\documentclass[11pt]{article}

%% Sample LaTeX for CS427/519
%% Mike Rosulek
%% last update 2017-01-23

\usepackage{xspace,graphicx,amsmath,amssymb,xcolor}

%% operators

\newcommand{\pct}{\mathbin{\%}}
% makes ":=" aligned better
\usepackage{mathtools}
\mathtoolsset{centercolon}

% indistinguishability operator
% http://tex.stackexchange.com/questions/22168/triple-approx-and-triple-approx-with-a-straight-middle-line
\newcommand{\indist}{  \mathrel{\vcenter{\offinterlineskip
  \hbox{$\sim$}\vskip-.35ex\hbox{$\sim$}\vskip-.35ex\hbox{$\sim$}}}}
\renewcommand{\cong}{\indist}



\newcommand{\K}{\mathcal{K}}
\newcommand{\M}{\mathcal{M}}
\newcommand{\C}{\mathcal{C}}
\newcommand{\Z}{\mathbb{Z}}
\newcommand{\A}{\mathcal{A}}
\newcommand{\T}{\mathcal{T}}
\newcommand{\D}{\mathcal{D}}

\newcommand{\Enc}{\text{\sf Enc}}
\newcommand{\Dec}{\text{\sf Dec}}
\newcommand{\KeyGen}{\text{\sf KeyGen}}
\newcommand{\eavs}{\text{EAVESDROP}}

% fancy script L
\usepackage[mathscr]{euscript}
\renewcommand{\L}{\ensuremath{\mathscr{L}}\xspace}
\newcommand{\lib}[1]{\ensuremath{\L_{\textsf{#1}}}\xspace}


\newcommand{\myterm}[1]{\ensuremath{\text{#1}}\xspace}
\newcommand{\bias}{\myterm{bias}}
\newcommand{\link}{\diamond}
\newcommand{\subname}[1]{\ensuremath{\textsc{#1}}\xspace}



%% colors
\definecolor{highlightcolor}{HTML}{F5F5A4}
\definecolor{highlighttextcolor}{HTML}{000000}
\definecolor{bitcolor}{HTML}{a91616}


%%% boxes for writing libraries/constructions
\usepackage{varwidth}

\newcommand{\codebox}[1]{%
        \begin{varwidth}{\linewidth}%
        \begin{tabbing}%
            ~~~\=\quad\=\quad\=\quad\=\kill % initialize tabstops
            #1
        \end{tabbing}%
        \end{varwidth}%
}
\newcommand{\titlecodebox}[2]{%
    \fboxsep=0pt%
    \fcolorbox{black}{black!10}{%
        \begin{varwidth}{\linewidth}%
        \centering%
        \fboxsep=3pt%
        \colorbox{black!10}{#1} \\
        \colorbox{white}{\codebox{#2}}%
        \end{varwidth}%
    }
}
\newcommand{\fcodebox}[1]{%
    \framebox{\codebox{#1}}%
}
\newcommand{\hlcodebox}[1]{%
    \fcolorbox{black}{highlightcolor}{\codebox{#1}}%
}
\newcommand{\hltitlecodebox}[2]{%
    \fboxsep=0pt%
    \fcolorbox{black}{black!15!highlightcolor}{%
        \begin{varwidth}{\linewidth}%
        \centering%
        \fboxsep=3pt%
        \colorbox{black!15!highlightcolor}{\color{highlighttextcolor}#1} \\
        \colorbox{highlightcolor}{\color{highlighttextcolor}\codebox{#2}}%
        \end{varwidth}%
    }
}


%% highlighting
\newcommand{\basehighlight}[1]{\colorbox{highlightcolor}{\color{highlighttextcolor}#1}}
\newcommand{\mathhighlight}[1]{\basehighlight{$#1$}}
\newcommand{\highlight}[1]{\raisebox{0pt}[-\fboxsep][-\fboxsep]{\basehighlight{#1}}}
\newcommand{\highlightline}[1]{%\raisebox{0pt}[-\fboxsep][-\fboxsep]{
    \hspace*{-\fboxsep}\basehighlight{#1}%
%}
}
\usepackage{soul}
\newcommand{\hlyellow}[1]{\sethlcolor{yellow}\hl{#1}}
\newcommand{\hlorange}[1]{\sethlcolor{orange}\hl{#1}}
\newcommand{\hlpink}[1]{\sethlcolor{pink}\hl{#1}}
\newcommand{\hlgreen}[1]{\sethlcolor{green}\hl{#1}}
\newcommand{\hlblue}[1]{\sethlcolor{cyan}\hl{#1}}
\newcommand{\hlpurple}[1]{\sethlcolor{magenta}\hl{#1}}
\newcommand{\hlgrey}[1]{\sethlcolor{lightgray}\hl{#1}}
\newcommand{\hlred}[1]{\sethlcolor{red}\hl{#1}}

%% bits
\newcommand{\bit}[1]{\textcolor{bitcolor}{\texttt{\upshape #1}}}
\newcommand{\bits}{\{\bit0,\bit1\}}

\usepackage[a4paper, total={6in, 8in}]{geometry} % Wide margins
\usepackage{enumitem}
\usepackage[english]{babel}

\usepackage[colorlinks=true, allcolors=blue]{hyperref} %Keep as last imported package, idk why

\begin{document}

\section*{Chapter 2}
\subsection*{Section Example 2.3}
\verb/se2_3Ots/

\subsection*{Homework 2 Problem 1}
\verb/hw2_1Ots/

Consider a variant of one-time pad where we avoid choosing the all-zeroes key. 
The modified KeyGen algorithm can be written as:
\[
    \fcodebox{
        \underline{KeyGen:}\\
        \> do \\
        \>\> $k \gets \bits^\lambda$\\
        \> until $k \ne 0^\lambda$\\
        \> return k
    }
\]
Hence $k$ is uniformly distributed over the \emph{set of all nonzero strings} 
of length $\lambda$. The Enc and Dec algorithms are the same as normal one-time pad.  
Formally show that this new encryption scheme does not satisfy one-time secrecy. 
Explicitly state the libraries that are relevant for this problem; write a 
calling program; derive the relevant output probabilities.

\section*{Chapter 5}
\subsection*{Homework 5 Problem 1}
\verb/hw5_1G/

Let $G : \bits^\lambda \implies \bits^{3\lambda}$ be a secure lengh-
\textbf{tripling} PRG. For each function below, state whether it is also a 
secure PRG. If the function is a secure PRG, give a proof. If not, then 
describe a successful distinguisher and explicitly compute its advantage. 
\begin{enumerate}[label=(\alph*)]
    \begin{minipage}{.3\linewidth}
        \item
        \framebox{
            \codebox{
                \underline{$H(s):$}\\
                \> $x := G(s)$\\
                \> $y := G(\bit0^\lambda)$\\
                \> return $x||y$
            }
        }
    \end{minipage}
    \begin{minipage}{.3\linewidth}
        \item 
        \framebox{
            \codebox{
                \underline{$H(s):$}\\
                \> $x := G(s)$\\
                \> $y := G(\bit0^\lambda)$\\
                \> return $x \oplus y$\\
            }
        }
    \end{minipage}
    \begin{minipage}{.3\textwidth}
        \item
        \framebox{
            \codebox{
                \underline{$H(s):$}\\
                \> $x||y||z := G(s)$\\
                \> $w := G(x)$\\
                \> return $x||y||z||w$
            }
        }
    \end{minipage}
\end{enumerate}

\section*{Chapter 6}
\subsection*{Homework 6 Problem 1}
\verb/hw6_1Prg/

Let $F$ be a secure PRF with $in = out = \lambda$. Define the 
following function: 
\[
    F'(k,m) = F(k,m) || F(k,F(k,m))
\]
Show that $F'$ is \textbf{not} a secure PRF.

\subsection*{Homework 6 Problem 2}
\verb/hw6_2Prg/

Show that a 2-round keyed Feistel cipher \textbf{cannot} be a secure PRP, no matter what its round
functions are. Your attack should work without knowing the round keys, and it should
work even with different (independent) round keys.
\\
\emph{Hint:} A successful attack requires two queries.

\section*{Chapter 7}
\subsection*{Homework 7 Problem 2}
\verb/hw7_2Cpa/

Let $F$ be a secure PRP with blocklength $\lambda$. Show that the following construction does not
have CPA/CPA\$ security:

\begin{center}
    \framebox{
        \codebox{
            \underline{Enc($k,m$):}\\
            \> $s_1 \gets \bits^\lambda$\\
            \> $s_2 := s_1 \oplus m$\\
            \> $x:= F(k,s_1)$\\
            \> $y:= F(k,s_2)$\\
            \> return $(x,y)$
        }
    }
\end{center}

\end{document}
